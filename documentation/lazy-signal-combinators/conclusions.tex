\section{Conclusions}

\texttt{SOUNDLAB}---even in its immature state---presents an opportunity
to explore abstract signal synthesis from scratch for engineers and artist
alike. Its simplicity encourages hacking and eases understanding. While
many complex problems surrounding signal synthesis remain unsolved, its
lazy combinatorial approach forms a powerful and extensible framework
capable of implementing classic as well as uncharted sound synthesis
features.

The demonstrated approach proved to be especially suited to exploratory
sound engineering. Ad-hoc signal pipelines can be built quickly in a
declarative way, encouraging re-usability and creativity. In comparison to
other tools in the domain the line between using and extending the system
is blurry. Where \textit{Csound} lets the user declaratively configure
instruments and controls using \textit{XML}, \texttt{SOUNDLAB} emphasizes
the user to use its built-in primitives and all of \textit{Common Lisp}
to stack layers of signal sources and modulators on top of each other.  
When compared to \textit{Overtone}---a \textit{Clojure} front-end to the
\textit{SuperCollider} audio system---\texttt{SOUNDLAB}'s back-end
independency and simplicity make it seem more suited for exploration and
hacking. Its core concepts are few and simple and its codebase is tiny
and modular despite some advanced features like envelopes, musical scales
and tempo, a lowpass filter and many kinds of signal combinators being
implemented.

Many of \textit{Common Lisp's} idioms proved to be an ideal
fit for the domain of signal synthesis. Furthermore, embedding a signal
synthesis language in \textit{Common Lisp} provides the system with
unmatched agility. While the core approach is mainly built on top of
functional paradigms, extensions like signal subtype checking---as
mentioned in section 3.2---could be implemented using macros.

I personally had tons of fun building and playing with
\texttt{SOUND\-LAB}. I encourage everyone interested in computerized
music to dive into the source code and experiment with the system---it
really is that simple. Feedback and contributions are welcome!
