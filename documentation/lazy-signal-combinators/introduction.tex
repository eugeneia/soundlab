\section{Introduction}

The described approach is mainly inspired from experience gained by using
analogue sound synthesizers. While every analogue synthesizer has its own
unique sound based on the physical parts it is made of, most do share
their key concepts. Usually a limited number of oscillators generate
signals resembling---more or less---sine waves which are then modulated
by being combined with each other in different ways.

\texttt{SOUNDLAB}---an experimental implementation of the presented
approach---is designed to enable the user to explore ways of signal
combination. It does so by defining an embedded \textit{domain specific
language} which provides axioms that generate primitive signals and
axioms that combine arbitrary signals into new signals. Furthermore
\texttt{SOUNDLAB} allows the use of \textit{Common Lisp's} means of
abstraction to define compound signals and signal combinators. Primitive
as well as compound parts of the system form a homogeneous group of
objects defined by their shared interfaces, which grant the system power
and flexibility of a \textit{Lisp} system.

There are of course many free software implementations\footnote{See for
instance Csound (\texttt{http://www.csounds.com/}).} of signal synthesis
systems with programming language interfaces. However, most concentrate
on integration with graphical interface toolkits rather than powerful
languages and none of these integrate tightly with \textit{Common Lisp}
to reuse the powerful means of abstraction provided by it.

\texttt{SOUNDLAB} is free software licensed under the \textit{GNU AGPL}
and can be obtained at \texttt{http://mr.gy/software/soundlab/}.
