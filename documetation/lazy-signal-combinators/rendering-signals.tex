Before discussing signal synthesis, we must define ways for consuming the
synthesized signal as well as for verification of our results. Because
our domain is music, we need to be able to play back signals as sound.
Furthermore visualizing a signal can be useful for debugging since some
properties of a signal are better conceivied visually than
aurally\footnote{Still, verifying signals by means of hearing has top
priority}.

For both forms of presentation a technique called \textit{sampling} is
used --- which will not be described in detail here. All that is needed
to know for this approach, is that the sampling routine records a
sequence of linear amplitude values according to a timespan and a
function --- or signal --- which maps values of time to values of
amplitude. The resulting sequence resembles the kind of data that can be
fed into standard digital sound adapters or plotting applications.

\begin{figure}
\centering
\texttt{(FUNCTION ((FUNCTION (REAL) REAL) REAL) (VECTOR REAL))}
\figcaption{Approximate type of a sampling function.}
\end{figure}

\texttt{SOUNDLAB} derives its signal type from this rationale. It also
exports two functions \texttt{EXPORT-FUNCTION-WAVE} and
\texttt{EXPORT-FUNCTION-GRAPH} which generate standard \textit{RIFF/WAVE}
audio files and \textit{Gnuplot} compatible data files.

\begin{figure}
\centering
\texttt{(FUNCTION (REAL) REAL)}
\figcaption{Type of a signal.}
\end{figure}
