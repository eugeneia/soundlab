\subsection{Signals as functions}

As discussed in the previous section, functions seem to be a natural way
to model a signal. Furthermore signals as functions encourage lazy
operations without enforcing them --- which can later be useful for
aggressive optimizations.

\begin{verbatim}
;;; Type of a signal.

(FUNCTION (REAL) REAL)
\end{verbatim}

A cruical type of signal is the sine wave --- since in theory, all
signals are sums of sine waves. \textit{Common Lisp} provides us with a
sine function \texttt{SIN} which serves our purpose well. We could pass
\texttt{\#'SIN} to a sampling routine as is, which would produce a very
low frequency signal below the human hearing threshold. In order to
specify other frequencies a constructor \texttt{SINE} is defined which
accepts a freqency in Hz and returns the respective sine signal.

\begin{verbatim}
;;; Constructor for primitive sine signals.

(defun sine (frequency)
  (lambda (x) (sin (* 2 pi frequency x))))
\end{verbatim}

Additonally a constructor for chorded signals could be defined as a
function that takes two signals as arguments and returns a function that
sums and normalises them accoring to the boundaries we defined in the
previous section.

\begin{verbatim}
;;; Constructor for a chord of two signals.

(defun chord-2 (signal-1 signal-2)
  (lambda (x) (* (+ (funcall signal-1 x)
                    (funcall signal-2 x))
                 1/2)))
\end{verbatim}

The \texttt{CHORD-2} function demonstrates the important traits of
signals as functions. A new signal in form of an anonymous function is
being compiled whenever we call \texttt{CHORD-2}. Because the actual
processing of the arguments is postponed until sampling occurs,
operation on signals is cheap. Furthermore calls to \texttt{CHORD-2}
can be chained to create chords with an arbitrary number of voices.
