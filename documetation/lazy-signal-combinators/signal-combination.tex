\subsection{Signal combination}

As seen in the previous section, modeling signals as functions
enables us to write small, cheap and powerful signal combinators which
can be chained to arbitrary extent. When chosen carefully, a small set of
primitive combinators and signals can be used to create infinitely
complex sounds.

\begin{verbatim}
;;; Type of a signal combinator.

(FUNCTION (&REST (FUNCTION (REAL) REAL))
          (FUNCTION (REAL) REAL))
\end{verbatim}

While building \texttt{SOUNDLAB}, some primitives turned out to be
especially useful. \texttt{FLATLINE}---a constant signal
constructor---serves a simple but important purpose. It takes a number as
its only argument and returns a flat signal with a constant
amplitude. When passed to a signal combinator its purpose is usually to
scale combinations of signals. \texttt{ADD} is a general signal adder. It
takes an arbitrary number of signals and sums them. Likewise,
\texttt{MULTIPLY} multiplies signals. The \texttt{CHORD-2} combinator of
the previous section can be defined more generally using these primitives.

\begin{verbatim}
;;; Implementation of FLATLINE.

(defun flatline (amplitude)
  (lambda (x)
    (declare (ignore x))
    amplitude))
\end{verbatim}

\begin{verbatim}
;;; Generic implementation of CHORD.

(defun chord (&rest signals)
  (multiply (apply #'add signals)
            (flatline (/ 1 (length signals)))))
\end{verbatim}

Furthermore, using signals as arguments to operations where constants
would suffice whenever possible has proven to be feasible and powerful.
Whenever a component is being modeled that would be controlled by a knob
or fader in an analogue synthesizer, then its digital counterpart should
be controlled by a signal. Take for instance a signal combinator
\texttt{MIX*} whose purpose is to merge two signals---just like
\texttt{CHORD}---while additionally providing a way to control how much
each input signal amounts to the mixed signal. So what would have been a
\textit{Dry/Wet} knob on an analogue synthesizer becomes a signal in our
case. Our \texttt{MIX*} takes three signals as arguments, two to be mixed
and a third to control their amounts. For ease of implementation we also
introduce \texttt{SUBTRACT}---the counterpart to \texttt{ADD}.

\begin{verbatim}
;;; Implementation of MIX*.

(defun mix (signal-a signal-b ratio-signal)
  (add (multiply signal-a
                 (subtract (flatline 1) ratio-signal))
       (multiply signal-b
                 ratio-signal)))
\end{verbatim}

Staying within closure of the signal representation---that is trying
hard to define our operations on a uniform signal representation
only---grants the system a lot of power and flexibility. All of the
presented signal combinators can be plugged into each other without
restriction. As of now some care has to be taken to not produce signals
exceeding the defined boundaries---see \textit{Rendering signals}.
Additionally, some combinators make use of non-audible signals. For
instance \texttt{MIX*} expects \texttt{RATIO-SIGNAL} to return values
ranging from zero to one and \texttt{MULTIPLY} is used in combination
with \texttt{FLATLINE} to moderate signals. Nevertheless, our few
examples can already be used to produce complex sounds. The code snippet
below works in \texttt{SOUNDLAB} as is and produces a rhythmically phasing
sound.

\begin{verbatim}
;;; Possible usage of the presented combinators.

(defun a-4 () (sine 440))
(defun a-5 () (sine 220))

;; Normalize a sine to 0-1 for use as RATIO-SIGNAL.
(defun sine-ratio ()
  (multiply (add (sine 1)
                 (flatline 1))
            (flatline 1/2)))

;; Produce a WAVE file.
(export-function-wave
  ;; A complex signal.
  (mix* (chord (a-4) (a-5))
        (multiply (a-4) (a-5))
        (sine-ratio))
  ;; Length of the sampling in seconds.
  4
  ;; Output file.
  #p"test.wav")
\end{verbatim}
