\section{Conclusions}

\texttt{SOUNDLAB}---even in its immature state---presents an opportunity
to explore abstract signal synthesis from scratch for engineers and artist
alike. Its simplicity encourages hacking and eases understanding. While
many complex problems surrounding signal synthesis remain unsolved, its
lazy combinatorial approach forms a powerful and extensible framework
capable of implementing classic as well as uncharted sound synthesis
features.

The demonstrated approach proved to be especially suited to exploratory
sound engineering. Ad-hoc signal pipelines can be built quickly in a
declarative way, encouraging re-usability and creativity. In comparison to
other tools in the domain the line between using and extending the system
is blurry. Many of \textit{Common Lisp's} idioms proved to be an ideal
fit for the domain of signal synthesis. Furthermore, embedding a signal
synthesis language in \textit{Common Lisp} provides the system with
unmatched agility.

I personally had tons of fun building and playing with \texttt{SOUNDLAB}.
I encourage everyone interested in computerized music to dive into the
source code and experiment with the system---it really is that simple.
Feedback and contributions are welcome!
