% THIS IS SIGPROC-SP.TEX - VERSION 3.1
% WORKS WITH V3.2SP OF ACM_PROC_ARTICLE-SP.CLS
% APRIL 2009
%
% It is an example file showing how to use the 'acm_proc_article-sp.cls' V3.2SP
% LaTeX2e document class file for Conference Proceedings submissions.
% ----------------------------------------------------------------------------------------------------------------
% This .tex file (and associated .cls V3.2SP) *DOES NOT* produce:
%       1) The Permission Statement
%       2) The Conference (location) Info information
%       3) The Copyright Line with ACM data
%       4) Page numbering
% ---------------------------------------------------------------------------------------------------------------
% It is an example which *does* use the .bib file (from which the .bbl file
% is produced).
% REMEMBER HOWEVER: After having produced the .bbl file,
% and prior to final submission,
% you need to 'insert'  your .bbl file into your source .tex file so as to provide
% ONE 'self-contained' source file.
%
% Questions regarding SIGS should be sent to
% Adrienne Griscti ---> griscti@acm.org
%
% Questions/suggestions regarding the guidelines, .tex and .cls files, etc. to
% Gerald Murray ---> murray@hq.acm.org
%
% For tracking purposes - this is V3.1SP - APRIL 2009

\documentclass{acm_proc_article-sp}
\usepackage[utf8x]{inputenc}
\usepackage[ngerman]{babel}

\begin{document}

\title{Lazy Signal Combinators with Common Lisp}

%
% You need the command \numberofauthors to handle the 'placement
% and alignment' of the authors beneath the title.
%
% For aesthetic reasons, we recommend 'three authors at a time'
% i.e. three 'name/affiliation blocks' be placed beneath the title.
%
% NOTE: You are NOT restricted in how many 'rows' of
% "name/affiliations" may appear. We just ask that you restrict
% the number of 'columns' to three.
%
% Because of the available 'opening page real-estate'
% we ask you to refrain from putting more than six authors
% (two rows with three columns) beneath the article title.
% More than six makes the first-page appear very cluttered indeed.
%
% Use the \alignauthor commands to handle the names
% and affiliations for an 'aesthetic maximum' of six authors.
% Add names, affiliations, addresses for
% the seventh etc. author(s) as the argument for the
% \additionalauthors command.
% These 'additional authors' will be output/set for you
% without further effort on your part as the last section in
% the body of your article BEFORE References or any Appendices.

\numberofauthors{1} %  in this sample file, there are a *total*
% of EIGHT authors. SIX appear on the 'first-page' (for formatting
% reasons) and the remaining two appear in the \additionalauthors section.
%
\author{
% You can go ahead and credit any number of authors here,
% e.g. one 'row of three' or two rows (consisting of one row of three
% and a second row of one, two or three).
%
% The command \alignauthor (no curly braces needed) should
% precede each author name, affiliation/snail-mail address and
% e-mail address. Additionally, tag each line of
% affiliation/address with \affaddr, and tag the
% e-mail address with \email.
%
% 1st. author
\alignauthor
Max Rottenkolber\\
       \affaddr{Rottenkolber Software Engineering}\\
       \affaddr{Karlstraße 15}\\
       \affaddr{53115 Bonn, Germany}\\
       \email{max@mr.gy}
}

\maketitle

\begin{abstract}

This demonstration explores an intuitive approach to signal synthesis
using \textit{Higher-order functions} to compute \textit{lazy}
represetations of signals. Given a few primitives which adhere to said
representation, ad hoc combinations of signals can be formed to be used
to synthesize new signals in a natural way. Results from an experimetal
implementation of this approach in an embedded \footnote{Embedded in
\textit{Common Lisp}, that is.} signal synthesis language called
\texttt{SOUNDLAB} prove the approach to be powerful and extensible.


\end{abstract}

% A category with the (minimum) three required fields
\category{J.5 }{Arts and Humanities}{Performing Arts}
%A category including the fourth, optional field follows...
\category{\\
H.5.5}{Information Systems}{Information Interfaces and Presentation}[Sound and Music Computing]

\terms{Demonstration}

\keywords{Signal synthesis, combinatory higher-order functions, Common
  Lisp} % NOT required for Proceedings

\section{Introduction}

The described approach is mainly inspired from experience gained by using
analogue sound synthesizers. While every analogue synthesizer has its own
unique sound based on the physical parts it is made of, most do share
their key concepts. Usually a limited number of oscillators generate
signals resembling---more or less---sine waves which are then modulated
by being combined with each other in different ways.

\texttt{SOUNDLAB}---an experimental implementation of the presented
approach---is designed to enable the user to explore ways of signal
combination. It does so by defining an embedded \textit{domain specific
language} which provides axioms that generate primitive signals and
axioms that combine arbitrary signals into new signals. Furthermore
\texttt{SOUNDLAB} allows the use of \textit{Common Lisp's} means of
abstraction to define compound signals and signal combinators. Primitive
as well as compound parts of the system form a homogeneous group of
objects defined by their shared interfaces, which grant the system power
and flexibility of a \textit{Lisp} system.

There are of course many free software implementations\footnote{See for
instance Csound (\texttt{http://www.csounds.com/}).} of signal synthesis
systems with programming language interfaces. However, most concentrate
on integration with graphical interface toolkits rather than powerful
languages and none of these integrate tightly with \textit{Common Lisp}
to reuse the powerful means of abstraction provided by it.

\texttt{SOUNDLAB} is free software licensed under the \textit{GNU AGPL}
and can be obtained at \texttt{http://mr.gy/software/soundlab/}.


Before discussing signal synthesis, we must define ways for consuming the
synthesized signal as well as for verification of our results. Because
our domain is music, we need to be able to play back signals as sound.
Furthermore visualizing a signal can be useful for debugging since some
properties of a signal are better conceivied visually than
aurally\footnote{Still, verifying signals by means of hearing has top
priority}.

For both forms of presentation a technique called \textit{sampling} is
used --- which will not be described in detail here. All that is needed
to know for this approach, is that the sampling routine records a
sequence of linear amplitude values according to a timespan and a
function --- or signal --- which maps values of time to values of
amplitude. The resulting sequence resembles the kind of data that can be
fed into standard digital sound adapters or plotting applications.

\begin{figure}
\centering
\texttt{(FUNCTION ((FUNCTION (REAL) REAL) REAL) (VECTOR REAL))}
\figcaption{Approximate type of a sampling function.}
\end{figure}

\texttt{SOUNDLAB} derives its signal type from this rationale. It also
exports two functions \texttt{EXPORT-FUNCTION-WAVE} and
\texttt{EXPORT-FUNCTION-GRAPH} which generate standard \textit{RIFF/WAVE}
audio files and \textit{Gnuplot} compatible data files.

\begin{figure}
\centering
\texttt{(FUNCTION (REAL) REAL)}
\figcaption{Type of a signal.}
\end{figure}


\section{Signal synthesis}

\subsection{Signals as functions}

As discussed in the previous section, functions seem to be a natural way
to model a signal. Furthermore signals as functions encourage lazy
operations without enforcing them --- which can later be useful for
aggressive optimizations.

\begin{verbatim}
;;; Type of a signal.

(FUNCTION (REAL) REAL)
\end{verbatim}

A cruical type of signal is the sine wave --- since in theory, all
signals are sums of sine waves. \textit{Common Lisp} provides us with a
sine function \texttt{SIN} which serves our purpose well. We could pass
\texttt{\#'SIN} to a sampling routine as is, which would produce a very
low frequency signal below the human hearing threshold. In order to
specify other frequencies a constructor \texttt{SINE} is defined which
accepts a freqency in Hz and returns the respective sine signal.

\begin{verbatim}
;;; Constructor for primitive sine signals.

(defun sine (frequency)
  (lambda (x) (sin (* 2 pi frequency x))))
\end{verbatim}

Additonally a constructor for chorded signals could be defined as a
function that takes two signals as arguments and returns a function that
sums and normalises them accoring to the boundaries we defined in the
previous section.

\begin{verbatim}
;;; Constructor for a chord of two signals.

(defun chord-2 (signal-1 signal-2)
  (lambda (x) (* (+ (funcall signal-1 x)
                    (funcall signal-2 x))
                 1/2)))
\end{verbatim}

The \texttt{CHORD-2} function demonstrates the important traits of
signals as functions. A new signal in form of an anonymous function is
being compiled whenever we call \texttt{CHORD-2}. Because the actual
processing of the arguments is postponed until sampling occurs,
operation on signals is cheap. Furthermore calls to \texttt{CHORD-2}
can be chained to create chords with an arbitrary number of voices.


\subsection{Signal combination}

As seen in the previous section, modeling signals as functions
enables us to write small, cheap and powerful signal combinators which
can be chained to arbitrary extent. When chosen carefully, a small set of
primitive combinators and signals can be used to create infinitely
complex sounds.

\begin{verbatim}
;;; Type of a signal combinator.

(FUNCTION (&REST (FUNCTION (REAL) REAL))
          (FUNCTION (REAL) REAL))
\end{verbatim}

While building \texttt{SOUNDLAB}, some primitives turned out to be
especially useful. \texttt{FLATLINE}---a constant signal
constructor---serves a simple but important purpose. It takes a number as
its only argument and returns a flat signal with a constant
amplitude. When passed to a signal combinator its purpose is usually to
scale combinations of signals. \texttt{ADD} is a general signal adder. It
takes an arbitrary number of signals and sums them. Likewise,
\texttt{MULTIPLY} multiplies signals. The \texttt{CHORD-2} combinator of
the previous section can be defined more generally using these primitives.  

\begin{verbatim}
;;; Implementation of FLATLINE.

(defun flatline (amplitude)
  (lambda (x)
    (declare (ignore x))
    amplitude))
\end{verbatim}

\begin{verbatim}
;;; Generic implementation of CHORD.

(defun chord (&rest signals)
  (multiply (apply #'add signals)
            (flatline (/ 1 (length signals)))))
\end{verbatim}

Note that---due to the normalization performed by \texttt{CHORD-2}---the
equivalent of \texttt{(chord a b c)} is

\begin{verbatim}
(chord-2 (chord-2 a b) (chord-2 c (flatline 1)))
\end{verbatim}

as opposed to 

\begin{verbatim}
(chord-2 (chord-2 a b) c)
\end{verbatim}

which would produce the chord of \texttt{C} and the chord of \texttt{A}
and \texttt{B} instead of the chord of \texttt{A}, \texttt{B} and
\texttt{C}.

Furthermore, using signals as arguments to operations where constants
would suffice whenever possible has proven to be feasible and powerful.
Whenever a component is being modeled that would be controlled by a knob
or fader in an analogue synthesizer, then its digital counterpart should
be controlled by a signal. Take for instance a signal combinator
\texttt{MIX*} whose purpose is to merge two signals---just like
\texttt{CHORD}---while additionally providing a way to control how much
each input signal amounts to the mixed signal. So what would have been a
\textit{Dry/Wet} knob on an analogue synthesizer becomes a signal in our
case. Our \texttt{MIX*} takes three signals as arguments, two to be mixed
and a third to control their amounts. For ease of implementation we also
introduce \texttt{SUBTRACT}---the counterpart to \texttt{ADD}.

\begin{verbatim}
;;; Implementation of MIX*.

(defun mix* (signal-a signal-b ratio-signal)
  (add (multiply signal-a
                 (subtract (flatline 1)
                           ratio-signal))
       (multiply signal-b
                 ratio-signal)))
\end{verbatim}

Staying within closure of the signal representation---that is trying
hard to define our operations on a uniform signal representation
only---grants the system a lot of power and flexibility. All of the
presented signal combinators can be plugged into each other without
restriction. As of now some care has to be taken to not produce signals
exceeding the defined boundaries---see \textit{Rendering signals}.
Additionally, some combinators make use of non-audible signals. For
instance \texttt{MIX*} expects \texttt{RATIO-SIGNAL} to return values
ranging from zero to one and \texttt{MULTIPLY} is used in combination
with \texttt{FLATLINE} to moderate signals. \texttt{SOUNDLAB} fails to
address the issue of having multiple informal subtypes of signals. As of
now the user has to refer to the documentation of a combinator to find
out if it expects certain constraints---as is the case with \texttt{MIX*}.
Nevertheless, our few examples can already be used to produce complex
sounds. The code snippet below works in \texttt{SOUNDLAB} as is and
produces a rhythmically phasing sound.

\begin{verbatim}
;;; Possible usage of the presented combinators.

(defun a-4 () (sine 440))
(defun a-5 () (sine 220))

;; Normalize a sine to 0-1 for use as RATIO-SIGNAL.
(defun sine-ratio ()
  (multiply (add (sine 1)
                 (flatline 1))
            (flatline 1/2)))

;; Produce a WAVE file.
(export-function-wave
  ;; A complex signal.
  (mix* (chord (a-4) (a-5))
        (multiply (a-4) (a-5))
        (sine-ratio))
  ;; Length of the sampling in seconds.
  4
  ;; Output file.
  #p"test.wav")
\end{verbatim}




%\section{The state of \secit{SOUNDLAB}}

As of the time of this writing \texttt{SOUNDLAB} consists of roughly 500
lines of source code. It depends on a minimal library for writing
\textit{WAVE} files and is written entirely in \textit{Common Lisp}. The
source code is fairly well documented and frugal.

While being compact \texttt{SOUNDLAB} provides basic routines for working
with western notes and tempo, a few primitive waveforms, \textit{ADSR}
envelopes with customizable slopes and the ability to form arbitrary
waveforms from envelopes, a good handfull of signal combinators and last
but not least an experimental lowpass filter. A stable API is nowhere
near in sight but some trends in design are becoming clear.

On the roadmap are classic sound synthesis features like resonance,
routines for importing signals from \textit{WAVE} files and many small
but essential details like bezier curved slopes for envelopes.


%\section{Conclusions}

\texttt{SOUNDLAB}---even in its immature state---presents an opportunity
to explore abstract signal synthesis from scratch for engineers and artist
alike. Its simplicity encourages hacking and eases understanding. While
many complex problems surrounding signal synthesis remain unsolved, its
lazy combinatorial approach forms a powerful and extensible framework
capable of implementing classic as well as uncharted sound synthesis
features.

The demonstrated approach proved to be especially suited to exploratory
sound engineering. Ad-hoc signal pipelines can be built quickly in a
declarative way, encouraging re-usability and creativity. In comparison to
other tools in the domain the line between using and extending the system
is blurry. Where \textit{Csound} lets the user declaratively configure
instruments and controls using \textit{XML}, \texttt{SOUNDLAB} emphasizes
the user to use its built-in primitives and all of \textit{Common Lisp}
to stack layers of signal sources and modulators on top of each other.  
When compared to \textit{Overtone}---a \textit{Clojure} front-end to the
\textit{SuperCollider} audio system---\texttt{SOUNDLAB}'s back-end
independency and simplicity make it seem more suited for exploration and
hacking. Its core concepts are few and simple and its codebase is tiny
and modular despite some advanced features like envelopes, musical scales
and tempo, a lowpass filter and many kinds of signal combinators being
implemented.

Many of \textit{Common Lisp's} idioms proved to be an ideal
fit for the domain of signal synthesis. Furthermore, embedding a signal
synthesis language in \textit{Common Lisp} provides the system with
unmatched agility. While the core approach is mainly built on top of
functional paradigms, extensions like signal subtype checking---as
mentioned in section 3.2---could be implemented using macros.

I personally had tons of fun building and playing with
\texttt{SOUND\-LAB}. I encourage everyone interested in computerized
music to dive into the source code and experiment with the system---it
really is that simple. Feedback and contributions are welcome!


%
% The following two commands are all you need in the
% initial runs of your .tex file to
% produce the bibliography for the citations in your paper.
\bibliographystyle{abbrv}
\bibliography{sigproc}  % sigproc.bib is the name of the Bibliography in this case
% You must have a proper ".bib" file
%  and remember to run:
% latex bibtex latex latex
% to resolve all references
%
% ACM needs 'a single self-contained file'!

\balancecolumns

% That's all folks!
\end{document}
